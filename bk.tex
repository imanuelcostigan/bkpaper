\documentclass{tufte-handout}

%\geometry{showframe}% for debugging purposes -- displays the margins

\usepackage{amsmath}

% More math symbols
\usepackage{amssymb}

% Set up the images/graphics package
\usepackage{graphicx}
\setkeys{Gin}{width=\linewidth,totalheight=\textheight,keepaspectratio}
\graphicspath{{graphics/}}

\title{A review of Burgard \& Kjaer's Funding costs, funding strategies}
\author{Imanuel Costigan, Model Risk, Westpac Banking Corporation}
\date{8 January 2015}  % if the \date{} command is left out, the current date will be used

% The following package makes prettier tables.  We're all about the bling!
\usepackage{booktabs}

% The units package provides nice, non-stacked fractions and better spacing
% for units.
\usepackage{units}

% The fancyvrb package lets us customize the formatting of verbatim
% environments.  We use a slightly smaller font
\usepackage{fancyvrb}
\fvset{fontsize=\normalsize}

% Small sections of multiple columns
\usepackage{multicol}

% Provides paragraphs of dummy text
\usepackage{lipsum}

% Provide robust multi-line equation environment IEEEeqnarray
\usepackage[retainorgcmds]{IEEEtrantools}

\usepackage{xcolor}

% These commands are used to pretty-print LaTeX commands
\newcommand{\doccmd}[1]{\texttt{\textbackslash#1}}% command name -- adds backslash automatically
\newcommand{\docopt}[1]{\ensuremath{\langle}\textrm{\textit{#1}}\ensuremath{\rangle}}% optional command argument
\newcommand{\docarg}[1]{\textrm{\textit{#1}}}% (required) command argument
\newenvironment{docspec}{\begin{quote}\noindent}{\end{quote}}% command specification environment
\newcommand{\docenv}[1]{\textsf{#1}}% environment name
\newcommand{\docpkg}[1]{\texttt{#1}}% package name
\newcommand{\doccls}[1]{\texttt{#1}}% document class name
\newcommand{\docclsopt}[1]{\texttt{#1}}% document class option name

\begin{document}

\maketitle% this prints the handout title, author, and date

\begin{abstract}
\noindent Burgard \& Kjaer (B\&K) \cite{bkfunding2013up} develop a derivative
pricing PDE based on a hedging strategy that eliminates market, counterparty
credit risk, but which does not necessarily eliminate all own credit risk. I\sidenote{Disclaimer: This note represents the views of the author alone, and not necessarily the views of Westpac Banking Corporation.}
will expand on their derivation of a derivative pricing PDE and correct an error
in it.
\end{abstract}

\section{Correcting the B\&K PDE}\label{sec:correction}

\subsection{Background}

A derivative is exposed to the three types of risks:

\begin{enumerate}
\item Market risk associated with the underlying asset $S$
\item Credit risk associated with the default of the counterparty $C$ before the
derivative's termination date.
\item Credit risk associated with the default of the issuer (say a bank) $B$
before the derivative's termination date.
\end{enumerate}

\subsection{Hedging strategy}\label{sec:dynhedging}

We can specify a self-financing, replicating\sidenote{Replicating except
perhaps at bank default} hedging strategy associated with the
derivative as seen from the perspective of $B$:

\begin{enumerate}
\item $\delta$ units of the asset $S$ financed by $\beta_S$ units of cash
from the repo market.
\item $\alpha_C$ units of the zero-recovery bond $P_C$ issued by $C$ financed
by $\beta_C$ units of cash from the repo market.
\item $\alpha_1$ and $\alpha_2$ units of the bonds $P_1$ and $P_2$ issued
by $B$ which have the recovery rates $R_1$ and $R_2$ respectively.
These bonds are bought back/issued with the hedging strategy's cash
surplus/shortfall.
\end{enumerate}

If the derivative belongs to a netting setting defined by a Credit Support
Annex (CSA), then the derivative will be supported by $X$ units of cash in
a collateral pool where $X>0$ when the derivative is in-the-money to the bank.
The bank can use $X$ to fund the cash needs of the hedging strategy if
the collateral can be rehypothecated. Let use denote the hedging strategy's value
proceess inclusive of collateral by $\Pi$. Then,

\begin{eqnarray}
  \Pi & = & \delta S + \beta_S + \alpha_C P_C + \beta_C +
          \alpha_1 P_1 + \alpha_2 P_2 - X \label{eq:pivalue} \\
  d\Pi & = & \delta dS + d\beta_S + \alpha_C dP_C + d\beta_C +
          \alpha_1 dP_1 + \alpha_2 dP_2 - dX \label{eq:pidelta}
\end{eqnarray}

The funding strategy of the asset $S$ and the bond $P_C$ outlined above can be
represented as:

\begin{eqnarray}
  \delta S + \beta_S & = & 0 \label{eq:financeS} \\
  \alpha_C P_C + \beta_C & = & 0 \label{eq:financePC}
\end{eqnarray}

Substituting (\ref{eq:financeS}) and (\ref{eq:financePC}) into (\ref{eq:pivalue}),
and using the fact that the hedging strategy replicates the derivative value
process we find that:

\begin{eqnarray*}
  \Pi & = & \alpha_1 P_1 + \alpha_2 P_2 - X \\
      & = & -\hat{V}
\end{eqnarray*}

As a result we find the following funding constraint:

\begin{equation}\label{eq:fundingconstraint}
  \hat{V} - X + \alpha_1 P_1 + \alpha_2 P_2 = 0
\end{equation}

\subsection{Dynamics}\label{sec:dynamics}

B\&K set out the following dynamics for elements of the hedging strategy:

\begin{eqnarray}
  dS       & = & \mu S dt + \sigma S dW \label{eq:dynstart}\\
  d\beta_S & = & \delta (\gamma_S - q_S) S dt \\
  dP_C     & = & r_C P_C dt - P_C dJ_C \\
  d\beta_C & = &  -\alpha_C q_C P_C dt \\
  dP_i     & = & r_i P_i dt - \bar{R}_i P_i dJ_B \quad i = 1, 2\\
  dX       & = & -r_X X dt \label{eq:dynend}
\end{eqnarray}

Note that:

\begin{enumerate}

\item $\gamma_S$ and $q_S$ denote the dividend income from and financing rate
for $S$ respectively;
\item $q_C$ denotes the financing rate for $P_C$;
\item $J_C$ and $J_B$ are the independent jump processes denoting the default
of $C$ and $B$ respectively;
\item B\&K assume that there is zero basis between $r_1$ and $r_2$ so that
$r_i = r + \hat{R}_i\lambda_B$;
\item $\bar{R}_i$ denotes the loss on the (pre)-default value of the bond $P_i$
so that $\bar{R}_i = 1 - R_i$\sidenote{$R_C = 0$ by definition}; and
\item $r_X$ denotes the rate accruing on the posted collateral $X$.
\end{enumerate}

When we substitute the dynamics of (\ref{eq:dynstart}) -- (\ref{eq:dynend}) into
(\ref{eq:pidelta}) and collecting the risk terms associated with $dS$, $dJ_B$ and
$dJ_C$ as well as the drift terms $dt$ we find that:

\begin{IEEEeqnarray}{rCl}
  d\Pi & = & \delta dS + d\beta_S + \alpha_C dP_C + d\beta_C +
          \alpha_1 dP_1 + \alpha_2 dP_2 - dX \nonumber\\
      & = & \delta dS +
               \delta (\gamma_S - q_S) S dt +
               \alpha_C (r_C P_C dt - P_C dJ_C) -\alpha_C q_C P_C dt \nonumber\\
      & & +\: \alpha_1 (r_1 P_1 dt - \bar{R}_1 P_1 dJ_B) +
              \alpha_2 (r_2 P_2 dt - \bar{R}_2 P_2 dJ_B) \mathbin{\color{red}+} r_X X dt \nonumber\\
      & = & \delta dS - \alpha_C P_C dJ_C -
              (\alpha_1 \bar{R}_1 P_1 + \alpha_2 \bar{R}_2 P_2) dJ_B \nonumber\\
      & & +\: \left(\delta (\gamma_S - q_S) S + (r_C - q_C) \alpha_C P_C
              + \alpha_1 r_1 P_1 + \alpha_2 r_2 P_2\right.\nonumber \\
      & & \quad \left. {} \mathbin{\color{red}+} r_X X\right) dt
      \label{eq:pideltawithdynamics}
\end{IEEEeqnarray}
as $dX = -r_X X dt$ (as per (\ref{eq:dynend})). This is different to the net position expression given by B\&K in equation (47)
of their working paper\cite{bkfunding2013up} (with the difference highlighted in red).

Then using the assumption that $r_1$ and $r_2$ have no basis (see the notes
above) and denoting $P = \alpha_1 P_1 + \alpha_2 P_2$ and
$P_D = \alpha R_1 P_1 + \alpha R_2 P_2$ we find that:

\begin{IEEEeqnarray}{rCl}
  \alpha_1 r_1 P_1 + \alpha_2 r_2 P_2 & = & \alpha_1 (r + \hat{R}_1\lambda_B) P_1 +
        \alpha_2 (r + \hat{R}_2\lambda_B) P_2 \nonumber \\
        & = & r P + \lambda_B(P - P_D) \label{eq:p12drifts}
\end{IEEEeqnarray}

Then, substituting (\ref{eq:p12drifts}) into (\ref{eq:pideltawithdynamics}), we
find that:

\begin{IEEEeqnarray}{rCl}
  d\Pi & = & \delta dS - \alpha_C P_C dJ_C - (P - P_D) dJ_B \nonumber\\
      & & +\: \left(\delta (\gamma_S - q_S) S + (r_C - q_C) \alpha_C P_C \right.\nonumber\\
      & & \quad \left. {} + r P + \lambda_B(P - P_D) + r_X X\right) dt \label{eq:pideltawithdynamics2}
\end{IEEEeqnarray}

\subsection{PDE}

As in B\&K, the PDE associated with the derivative's value when exposed to
counterparty and bank risk:

\begin{IEEEeqnarray}{rCl}
  d\hat{V} & = & \partial_t \hat{V} dt + \partial_S \hat{V} dS +
    \frac{1}{2}\sigma^2 S^2 \partial^2\hat{V_S} dt + \Delta\hat{V}_B dJ_B +
    \Delta\hat{V}_C dJ_C \nonumber\\
    & = & \partial_S \hat{V} dS + \Delta\hat{V}_C dJ_C + \Delta\hat{V}_B dJ_B +
    \left(\partial_t \hat{V} + \frac{1}{2}\sigma^2 S^2 \partial^2_S\hat{V}\right)dt\label{eq:ito}
\end{IEEEeqnarray}
where $\Delta \hat{V}_B = g_B - \hat{V}$, $\Delta \hat{V}_C = g_C - \hat{V}$
and $g_B$ and $g_C$ denote the close-out values of the derivative when $B$ and
$C$ default\sidenote{See B\&Ks paper for what they term "regular" close-out
values}.

\subsection{Net position}\label{sec:netposition}

Let's now consider the change in the bank's net value of the hedging
portfolio (\ref{eq:pideltawithdynamics2}) and the derivative (\ref{eq:ito}):

\begin{IEEEeqnarray}{rCl}
  d\hat{V} + d\Pi & = & (\partial_S \hat{V} + \delta) dS +
    (\Delta\hat{V}_C -\alpha_C P_C)dJ_C + (\Delta\hat{V}_B -(P - P_D))dJ_B\nonumber\\
    & & \:+ \left(\partial_t \hat{V} + \frac{1}{2}\sigma^2 S^2 \partial^2_S\hat{V} + \delta (\gamma_S - q_S) S \right.\nonumber\\
    & & \quad \left. {} + (r_C - q_C) \alpha_C P_C + r P + \lambda_B(P - P_D)  + r_X X\right.\bigg) dt \label{eq:netposition} \\
    & \triangleq & \chi dS + Y_C dJ_C + Y_B dJ_B + Z dt
\end{IEEEeqnarray}

Suppose that the bank would like to perfectly hedge the P\&L due to variations
in $S$ or $J_C$. Then from (\ref{eq:netposition}) we can see that this can be
achieved by holding the following units of $S$ and $P_C$ respectively.
\sidenote{As there may be restrictions on the bank's hedging strategy associated with its
own default, we do not, for the moment, assume that the the bank wishes to hedge
its own default risk}

\begin{IEEEeqnarray}{rCl}
  \delta & = &  -\partial_S \hat{V} \label{eq:s_hedge_ratio}\\
  \alpha_C & = & \frac{\Delta\hat{V}_C}{P_C} \triangleq
                 \frac{g_C - \hat{V}}{P_C} \label{eq:pc_hedge_ratio}
\end{IEEEeqnarray}

When the holdings of $S$ and $P_C$ conform to these
quantities, $X = Y = 0$, denoting $\lambda_C = r_C - q_C$ and
$\mathcal{A}_t\hat{V} = \frac{1}{2}\sigma^2 S^2 \partial^2_S\hat{V} -(\gamma_S - q_S) S \partial_S \hat{V}$
and using the funding constraint in (\ref{eq:fundingconstraint}) along with the
notation we introduced earlier for $P$ we find that:

\begin{IEEEeqnarray}{rCl}
  Z dt & = & \left(\partial_t \hat{V} + \frac{1}{2}\sigma^2 S^2 \partial^2_S\hat{V}
        -(\gamma_S - q_S) S \partial_S \hat{V} \right.\nonumber\\
    & & \quad \left. {} + (r_C - q_C) (g_C - \hat{V}) \right.\nonumber \\
    & & \quad \left. {} + r P + \lambda_B(P - P_D) + r_X X\right.\bigg) dt \nonumber\\
    & = & \left(\partial_t \hat{V} + \mathcal{A}_t\hat{V} + \lambda_C (g_C - \hat{V}) \right.\\
    & & \quad \left. {} + r (X - \hat{V}) + \lambda_B(X - \hat{V} + P_D) + r_X X\right) dt
\end{IEEEeqnarray}

And consequently, with $\epsilon_h = g_B + P_D - X$, we find that:

\begin{IEEEeqnarray}{rCl}
  d\hat{V} + d\Pi & = & (\Delta\hat{V}_B -(P - P_D))dJ_B \nonumber\\
    & & \:+ \left(\partial_t \hat{V} + \mathcal{A}_t\hat{V} + \lambda_C (g_C - \hat{V}) \right.\nonumber\\
    & & \quad \left. {} + r (X - \hat{V}) + \lambda_B(X - \hat{V} + P_D) + r_X X\right) dt \nonumber\\
    & = &  (g_B - \hat{V} - (X - \hat{V} - P_D))dJ_B \nonumber\\
    & & \:+ \left(\partial_t \hat{V} + \mathcal{A}_t\hat{V} + \lambda_C (g_C - \hat{V}) \right.\nonumber\\
    & & \quad \left. {} + r (X - \hat{V}) + \lambda_B(X - \hat{V} + P_D) + r_X X\right) dt \nonumber\\
    & = & \epsilon_h dJ_B + \left(\partial_t \hat{V} + \mathcal{A}_t\hat{V} + \lambda_C (g_C - \hat{V}) \right.\nonumber\\
    & & \quad \left. {} + r (X - \hat{V}) + \lambda_B(X - \hat{V} - P_D) + r_X X\right) dt \nonumber \\
    & = & \left(\partial_t \hat{V} + \mathcal{A}_t\hat{V} - (r + \lambda_C + \lambda_B) \hat{V} \right. \nonumber \\
    & & \quad \left. {} + \lambda_C g_C + (X - P_D)\lambda_B + (r + r_X) X  \right) + \epsilon_h dJ_B \nonumber\\
    & = & \left(\partial_t \hat{V} + \mathcal{A}_t\hat{V} - (r + \lambda_C + \lambda_B) \hat{V} \right. \nonumber \\
    & & \quad \left. {} + \lambda_C g_C + (g_B - \epsilon_h) \lambda_B + (r + r_X) X  \right) + \epsilon_h dJ_B  \nonumber\\
    & = & \left(\partial_t \hat{V} + \mathcal{A}_t\hat{V} - (r + \lambda_C + \lambda_B) \hat{V} \right. \nonumber \\
    & & \quad \left. {} + \lambda_C g_C + \lambda_B g_B + (r + r_X) X -\lambda_B\epsilon_h \right) + \epsilon_h dJ_B
\end{IEEEeqnarray}

For the hedging strategy to perfectly replicate the derivative, we need the
following conditions to hold:

\begin{IEEEeqnarray}{rCl}
  0 & = & \tilde{\epsilon}_h = g_B + P_D - X \\
  0 & = & \partial_t \hat{V} + \mathcal{A}_t\hat{V} - (r + \lambda_C + \lambda_B) \hat{V} \nonumber \\
    & & \quad + \lambda_C g_C + \lambda_B g_B + (r + r_X) X -\lambda_B\epsilon_h \label{eq:bkpdecorrect}
\end{IEEEeqnarray}

The difference between (\ref{eq:bkpdecorrect}) and the one presented by B\&K is
that the coefficient of $X$ is $+(r + r_X)$ rather than $-(r_X - r)$ (respectively) -
the latter which would appear to have a more natural explanation. Unfortunately,
the latter term does not follow from the specified hedging strategy.

\bibliography{bk}
\bibliographystyle{plainnat}



\end{document}
